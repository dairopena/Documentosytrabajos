% arara: clean: {
% arara: --> extensions:
% arara: --> ['log','aux','out','pytxcode','synctex.gz','toc',
% arara: --> 'bbl','bcf','blg','run.xml','lof','pdf']
% arara: --> }
% arara: pdflatex: {
% arara: --> shell: 1,
% arara: --> interaction: batchmode
% arara: --> }
% arara: pythontex: {
% arara: --> jobs: 4,
% arara: --> verbose: yes
% arara: --> }
% arara: pdflatex: {
% arara: --> shell: 1,
% arara: --> interaction: batchmode
% arara: --> }
% arara: clean: {
% arara: --> extensions:
% arara: --> ['log','aux','out','pytxcode','synctex.gz','toc',
% arara: --> 'bbl','bcf','blg','run.xml','lof']
% arara: --> }





\documentclass[a4paper]{article}
\usepackage[utf8]{inputenc}
\usepackage{amsmath, amsthm, amssymb, amsfonts}
\usepackage{graphicx}
\usepackage{multicol}
\usepackage[total={18cm,24cm},centering]{geometry}
\usepackage{tcolorbox}
\tcbuselibrary{theorems}
\usepackage[utf8]{inputenc}
\usepackage[colorlinks,citecolor=blue,linkcolor=blue,urlcolor=blue]{hyperref}
\usepackage{cancel}
\usepackage{array}
\usepackage{float}
\usepackage[spanish,activeacute,es-tabla]{babel}
\usepackage{graphicx}
\usepackage{titlesec}
\usepackage{epstopdf}
%\usepackage{tabularx}
\usepackage{mathpazo}
\usepackage{keyval,ifthen,latexsym,graphicx,moreverb}
\usepackage{txfonts}
\usepackage{svg}
\usepackage{times}
\usepackage{array, multirow, multicol}
\usepackage{xltabular}
\usepackage{booktabs}
\usepackage{subcaption}
\usepackage{comment}
\usepackage{pythontex}


\begin{document}
 %calcular las raíces del polinomio $p(x)=2x^6+5x^5+2x^2-3x+4=0$. 
    \begin{pycode}
import numpy as np
import sympy as sym
from sympy import *

# Definir las variables simbólicas
#definir los coeficientes del polinomio


n = 4

x, y, z = sym.symbols('x y z')

# Definir el polinomio
polinomio = x**5 + 3*x**2*y + 3*x*y**2 + y**3

# Factorizar el polinomio
polinomio_factorizado = sym.factor(polinomio)

\end{pycode}

el polinomio $p(x)=\py{latex(polinomio)}$ factorizado es $\py{latex(polinomio_factorizado)}$.

de igual manera para factorizar un polinomio con raices reales e imaginarias.

\begin{pycode}
from sympy import *

#definir el polinomio

x = symbols('x')
polinomio2= x**2 + 2*x + 1

#factorizar el polinomio

factores2 = factor(polinomio2)

# mostrar el polinomio factorizado con raices reales e imaginarias

for f in factores.args:
      print(f)
      raices = solve(f,x)
      for r in raices:
	if r.is_real:
		print(r)
	else:
		print(r.evalf())
\end{pycode}

las raíces del polinomio $p(x)=\py{latex(polinomio2)}$.

\end{document}






