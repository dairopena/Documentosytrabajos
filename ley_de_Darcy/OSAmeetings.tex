%%%%%%%%%%%%%%%%%%%%%%%%%%%%%%%%%%%%%%%%%%%%%%%%%%%%%%%
%                   File: OSAmeetings.tex             %
%                  Date: 29 Novemver 2018              %
%                                                     %
%     For preparing LaTeX manuscripts for submission  %
%       submission to OSA meetings and conferences    %
%                                                     %
%       (c) 2018 Optical Society of America           %
%%%%%%%%%%%%%%%%%%%%%%%%%%%%%%%%%%%%%%%%%%%%%%%%%%%%%%%

\documentclass[letterpaper,10pt]{article} 
%% if A4 paper needed, change letterpaper to A4

\usepackage{osameet3} %% use version 3 for proper copyright statement

%% provide authormark
\newcommand\authormark[1]{\textsuperscript{#1}}

%% standard packages and arguments should be modified as needed
\usepackage{amsmath,amssymb}
%\usepackage[colorlinks=true,bookmarks=false,citecolor=blue,urlcolor=blue]{hyperref}

\usepackage[utf8]{inputenc}
\usepackage[T1]{fontenc}
\usepackage{lmodern}
\usepackage{amsmath, amsthm, amssymb}
\usepackage{float}
\usepackage{minted}
\usepackage[spanish,activeacute,es-tabla]{babel}
\usepackage{csquotes}
%\usepackage{graphics}
\usepackage{graphicx}
\usepackage{epstopdf-base}
\usepackage{titlesec}
\usepackage{tabularx}
\usepackage{times}
\usepackage{booktabs}
\usepackage{mathtools}
\usepackage{commath}
%\usepackage{tcolorbox}
\renewcommand{\shorthandsspanish}{}
\usepackage{lscape}
\usepackage{array, multirow, multicol}
\usepackage{longtable}
\usepackage{booktabs}
\usepackage{cancel}
\usepackage{tikz,color}
\usepackage{tikz-3dplot}
\usepackage{txfonts}
\usepackage{booktabs}
\usepackage{subcaption}
\usepackage{txfonts}
\titleformat*{\subsection}{\it}

\usepackage[backend=biber,style=apa]{biblatex}

\addbibresource{Quinto _Informe.bib}
\usepackage[colorlinks,citecolor=blue,linkcolor=blue,urlcolor=blue]{hyperref}


\begin{document}

\title{Modelos Matemáticos Para Suelos Agrícolas, Ley de Darcy}


\author{Dairo Janover Peña Cardenas,\authormark{1} John Jairo Leal Gómez \authormark{2}}

\address{\authormark{1} Facultad de ingeniería y administración, Universidad Nacional de Colombia - Sede Palmira, Palmira Valle del Cauca , 2022.\\}
\address{\authormark{2,*} Departamento de Ciencias Básicas, Facultad de Ingeniería y Administración, Universidad Nacional de Colombia - Sede Palmira, Palmira Valle del Cauca, 2022. 
\email {jlealgom@unal.edu.co}}\\

%% email address is required



\begin{abstract}

Se realizó una revisión bibliográfica de los modelos matemáticos que permiten comprender los fenómenos físicos que ocurren dentro del suelo, algunos de ellos como la humedad y la filtración, y que se estudian con ecuaciones diferenciales parciales como la Ley de Darcy y ecuaciones derivadas.  Se presentan diversos trabajos en los cuales se incorporan éstas herramientas matemáticas para el estudio de los suelos agrícolas.

\end{abstract}

\section{Introducción}

\input{reseña}


\section{Conclusiones}
Después de estudiar las posibles aplicaciones de la Ley de Darcy y sus leyes derivadas se puede apreciar que los modelos matemáticos nos permiten comprender las interacciones físicas suelo-agua, e iniciar un camino hacia el modelamiento y la simulación de diferentes tipos de suelos agrícolas, lo que impactará los estudios en el agro en general, pues se incorporarán herramientas tecnológicas para el estudio de tales fenómenos.








\printbibliography



\end{document}
